\chapter*{Bibliographie}
\addcontentsline{toc}{chapter}{Bibliographie}
\lhead{}
\rhead{\textit{Bibliographie}}
\textbf{\underline{a. Bibliographie :}} Ouvrages et articles consultés lors de l’élaboration du rapport, classés par ordre alphabétique du nom de l’auteur, selon le modèle suivant :

[i] NOM\_AUTEUR1, NOM\_AUTEUR2,   "Titre de l’ouvrage" , lieu de publication, nom de l’éditeur, année de publication.

\textbf{Exemple :}

[1] REEVES, Hubert.  Bases de données relationnelles , Paris, Editions du seuil, 1988.

\hrulefill

\textbf{ \underline{b. Webographie :} } Sites Web visités lors de l’élaboration du projet, avec une brève description du thème consulté (une ou deux lignes au maximum) avec la date de la dernière visite.

\textbf{Exemple :}

[2] http://www.asp.net/ : Fondements du langage ASP.NET. DV : mai 2017

\hrulefill

\textbf{{\large Remarque :}} dans le texte du rapport, vous mettez des liens vers ces références.

\textbf{Exemple :} ….les bases de données ….. intégrité … requête [1].
\bibliographystyle{unsrt}
%\bibliography{biblio}
\newpage
