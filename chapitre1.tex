\chapter{Contexte et Problématique}
\label{chap:intro}

\section{Introduction Générale}
\label{sec:intro}
La prédiction des défauts logiciels est un sujet de grande importance dans l’ingénierie logicielle. Les défauts, qu’il s’agisse de bogues ou d’erreurs structurelles, peuvent provoquer des dysfonctionnements critiques dans les systèmes logiciels. Ces problèmes se traduisent par des coûts supplémentaires liés à la maintenance, une insatisfaction des utilisateurs et parfois des impacts désastreux sur les entreprises.

Avec l’évolution des technologies et la croissance des systèmes informatiques, le volume de données produites lors du développement logiciel explose. Pourtant, les méthodes traditionnelles de détection de défauts, comme les tests manuels et les revues de code, montrent des limites évidentes, notamment en termes de temps requis et de risque d’erreurs humaines. C’est dans ce contexte que le machine learning émerge comme une solution prometteuse. L’objectif est d’utiliser des algorithmes capables de prévoir les défauts en analysant les données et en identifiant des motifs complexes invisibles pour l’humain.

Les enjeux dans ce domaine incluent la nécessité d’améliorer continuellement la qualité logicielle et de minimiser les coûts associés aux erreurs imprévues. Cependant, des défis subsistent : comment garantir une collecte de données de bonne qualité, quels modèles sont les mieux adaptés, et comment réduire les risques de surapprentissage.

\section{Objectifs du Projet}
\label{sec:objectifs}
\subsection{Objectif général}
Développer un système basé sur le machine learning, utilisant des techniques de classification, pour prédire efficacement les défauts logiciels.

\subsection{Objectifs spécifiques}
\begin{itemize}
    \item Collecter et structurer des bases de données pertinentes, issues de référentiels de code ou de systèmes de gestion des tickets.
    \item Explorer plusieurs techniques de classification pour sélectionner les plus performantes.
    \item Optimiser, entraîner et valider les modèles sur des jeux de données diversifiés.
\end{itemize}

