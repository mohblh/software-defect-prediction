\chapter{Introduction à la Machine Learning}
\label{chap:ml}

\section{Qu’est-ce que le Machine Learning ?}
\label{sec:ml}
L’apprentissage automatique ( Machine Learning) est un domaine de l’intelligence artificielle (IA) qui vise à
permettre aux machines à partir d’apprendre à partir d’un grand volume de données généralement sous forme de base de données pour prendre des décisions, l’amélioration de performance et faire des prédictions sans être explicitement programmés. Son objectif fondamental est "d’apprendre à apprendre" aux ordinateurs et par la suite, à agir et réagir comme le font les humains, en améliorant leur mode d'apprentissage et leurs connaissances de façon autonome.


\section{Applications du Machine Learning}
\label{sec:applications}
Les applications du machine learning couvrent différents domaines :
\begin{itemize}
    \item Santé : Diagnostic basé sur des images médicales.
    \item Finances : Détection de fraudes bancaires.
    \item Industrie : Maintenance prédictive des machines.
    \item Technologie logicielle : Prédiction des défauts logiciels.
\end{itemize}

\section{Processus Typique d’un Projet de Machine Learning}
\label{sec:processus}
\begin{itemize}
    \item Collecte des données : Obtenir des données d’entrainement et de tests adapées au problème ciblé.
    \item Prétraitement des données : Nettoyer, transformer et sélectionner les caractéristiques pertinentes.
    \item Sélection des modèles : Tester divers algorithmes pour choisir le plus adéquat.
    \item Entraînement et validation : Optimiser les modèles et évaluer leurs performances.
    \item Déploiement : Intégrer le modèle pret à l’emploi dans un environnement réel.
\end{itemize}